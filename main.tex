\documentclass[sigconf,review]{acmart}

\usepackage{color}
\usepackage{multirow}
\usepackage{xcolor}
\usepackage{colortbl}
\usepackage{booktabs}
\usepackage{balance}
\usepackage{float}
\usepackage{enumitem}
\usepackage{tcolorbox}
\usepackage[colorinlistoftodos]{todonotes}
\usepackage{xspace}

\usepackage{tikz}
%\usepackage{lipsum}

\definecolor{gray50}{gray}{.5}
\definecolor{gray40}{gray}{.6}
\definecolor{gray30}{gray}{.7}
\definecolor{gray20}{gray}{.8}
\definecolor{gray10}{gray}{.9}
\definecolor{gray05}{gray}{.95}

\newlength\Linewidth
\def\findlength{\setlength\Linewidth\linewidth
	\addtolength\Linewidth{-4\fboxrule}
	\addtolength\Linewidth{-3\fboxsep}
}

\newcommand\fk[1]{{{\textcolor{red}{Foutse: }}\color{red}#1}}
\newcommand\ps[1]{{{\textcolor{red}{Pasquale: }}\color{red}#1}}
\newcommand\gc[1]{{{\textcolor{red}{Gemma: }}\color{red}#1}}

%%
%% \BibTeX command to typeset BibTeX logo in the docs
\AtBeginDocument{%
	\providecommand\BibTeX{{%
			\normalfont B\kern-0.5em{\scshape i\kern-0.25em b}\kern-0.8em\TeX}}}

%% Rights management information.  This information is sent to you
%% when you complete the rights form.  These commands have SAMPLE
%% values in them; it is your responsibility as an author to replace
%% the commands and values with those provided to you when you
%% complete the rights form.
\setcopyright{acmcopyright}
\copyrightyear{2020}
\acmYear{2020}
\acmDOI{10.1145/1122445.1122456}

%% These commands are for a PROCEEDINGS abstract or paper.
%\acmConference[ICSE 2020]{ICSE 2020: IEEE/ACM International Conference on Software Engineering}{May 23--29, 2020}{Seoul, South Korea}
%\acmBooktitle{ICSE 2020: IEEE/ACM International Conference on Software Engineering, May 23--29, 2020 - Seoul, South Korea}
%\acmPrice{15.00}
%\acmISBN{978-1-4503-9999-9/18/06}

\newcommand{\ie}{\emph{i.e.},\xspace}
\newcommand{\eg}{\emph{e.g.},\xspace}
\newcommand{\etc}{\emph{etc.}\xspace}
\newcommand{\etal}{\emph{et al.}\xspace}
\newcommand{\aka}{\emph{a.k.a.}\xspace}

\newenvironment{myquote}[1]%
  {\list{}{\leftmargin=#1\rightmargin=#1}\item[]}%
  {\endlist}

\begin{document}
	\title{MaLTeSQuE: Machine Learning Techniques \\
		for Software Quality Evaluation}
	\subtitle{A workshop proposal for ESEC/FSE 2020 -- \url{https://maltesque2019.github.io}}
	
	\author{Gemma Catolino}
	\affiliation{%
		\institution{Delft University of Technology}
		\city{Delft}
		\state{The Netherlands}
	}
	\email{g.catolino@tudelft.nl}
	
		\author{Foutse Khomh$^*$}
	\affiliation{%
		\institution{Polytechnique Montréal}
		\city{Montréal}
		\state{Canada}
	}
	\email{foutse.khomh@polymtl.ca}
	
	\author{Pasquale Salza}
	\affiliation{%
		\institution{University of Zurich}
		\city{Zurich}
		\state{Switzerland}
	}
	\email{salza@ifi.uzh.ch}
	
	%\renewcommand{\shortauthors}{}
	

\begin{abstract}
% Limit: 175 words
We propose the 5$^{th}$ edition of the workshop on Machine Learning Techniques for Software Quality Evaluation (MaLTeSQuE) to be collocated with the next edition of ESEC/FSE 2020.
We aim at creating an enthusiastic forum for researchers and practitioners where to present and discuss new ideas, trends and results concerning the application of machine learning methods for software quality evaluation, and the application of software engineering techniques to self-learning systems.
\end{abstract}

%%
%% Keywords. The author(s) should pick words that accurately describe
%% the work being presented. Separate the keywords with commas.
\keywords{Software Engineering, Machine Learning, Software Quality}

%%
%% This command processes the author and affiliation and title
%% information and builds the first part of the formatted document.
\maketitle

%!TEX root = main.tex
% !TeX spellcheck = en_US

\section{Introduction}
\label{sec:motivation}

The assessment of software quality is one of the most multifaceted (\eg structural, product, and quality) and subjective aspects of software engineering (since in many cases it is substantially based on expert judgment).
Such assessments can be performed at almost all the phases of software development (from project inception to maintenance) and at different levels of granularity (from source code to architecture).
However, human judgment is inherently biased by implicit, subjective criteria applied to the evaluation process, and its economical effectiveness is limited compared to automated or semi-automated approaches.
To this end, researchers are still looking for new and more effective methods for assessing various qualitative characteristics of software systems and the related processes.
In recent years, we have been observing a rising interest in adopting various approaches to exploiting machine learning and automated decision-making processes in several areas of software engineering.
The machine learning models and algorithms help to reduce effort and risk related to human judgment in favor of automated systems, which are able to make informed decisions based on available data and evaluated with objective criteria.
Therefore, the adoption of machine learning techniques seems to be one of the most promising ways to improve software quality evaluation.
Conversely, learning capabilities are increasingly embedded within software, including in critical domains such as automotive and health.
For this reason, the application of quality assurance techniques is required to ensure the reliable engineering of software systems based on machine learning.
After the successful past editions of the workshop on Machine Learning Techniques for Software Quality Evaluation (MaLTeSQuE), which have been held in:
\begin{enumerate}
    \item Klagenfurt (Austria) on February \nth{21}, 2017, collocated with SANER 2017,
    \item Campobasso (Italy) on March \nth{23} , 2018, collocated with SANER 2018, and
    \item Tallin (Estonia) on August \nth{27} 2019, collocated with ESEC/FSE 2019,
    \item Virtual Event, November  \nth{16} 2020, collocated with ESEC/FSE 2020,
\end{enumerate}
we propose a novel edition of the workshop.
We aim at reaching the same, or hopefully better result, the workshop had the last year more than of 40 participants.

%!TEX root = main.tex
\section{Objective and Topics}
\label{sec:objective}

In the following we describe the objective of the proposed workshop and the topics we envision for the next edition.

\subsection{Objective}
The aim of the workshop is to provide a forum for researchers and practitioners to present and discuss new ideas, trends and results concerning the application of machine learning methods for software quality evaluation, and the application of software engineering techniques to self-learning systems.
We expect that the workshop will help with:
\begin{enumerate}
    \item the validation of existing machine learning methods for software quality evaluation as well as their application to novel contexts,
    \item the effectiveness evaluation of machine learning methods, both compared to other automated approaches and the human judgement,
    \item the adaptation of machine learning approaches already used in other areas of science in the context of software quality,
    \item the design of new techniques to validate software based on machine learning, inspired by traditional software engineering techniques.
\end{enumerate}

\subsection{Topics}
Topics of interest include, but are not limited to:
\begin{itemize}
    \item application of machine-learning in software quality evaluation;\smallskip
    \item analysis of multi-source data;\smallskip
    \item knowledge acquisition from software repositories;\smallskip
    \item adoption and validation of machine learning models and algorithms in software quality;\smallskip
    \item decision support and analysis in software quality;\smallskip
    \item prediction models to support software quality evaluation;\smallskip
    \item validation and verification of systems, learning;\smallskip
    \item automated machine learning;\smallskip
    \item design of safety-critical learning software;\smallskip
    \item integration of learning systems in software ecosystems.
\end{itemize}
%!TEX root = main.tex
\section{Workshop Formatting}
\label{sec:format}

The workshop will follow a one-day format, consisting of 3 to 4 sessions, depending on the number of papers accepted for publication.
The workshop is intended to be highly interactive: for this reason, each accepted paper will have a maximum of 15/20 minutes for the presentation, followed by 10/15 minutes for questions and discussion.
We hope that the workshop will foster and promote collaboration, and there will be some time set aside to support this.
We also plan to invite a keynote speaker who will be selected as a kind of advertisement to raise the public profile of the workshop and attract more people to attend and submit their research.
It means that the person chosen will be a well-known personality in that particular topic.
Finally, we also plan to support wide dissemination of the accepted contributions as well as the participants’ discussion.
To this purpose, we will ask the contributors to share their presentation on our website and we aim at being highly present on social networks, such as Facebook and Twitter.
Thus, we plan to have a person responsible for publicity and dissemination of the contents during the entire workshop.

%!TEX root = main.tex
\section{Paper selection procedure}
\label{sec:submissions}

We are looking for two different types of papers; the first one, \ie technical papers, where we invite to submit original research (even at early stages of evaluation) on how machine learning and software quality assurance can support each other.
Papers must not exceed four 6 pages for the main text, including figures, tables, appendices, and references.

The second one, \ie presentation abstracts, which have to report the research results that have already been published, or that are ready to be submitted to a conference or a journal.
They will only be reviewed for relevance, and will not be included in the MaLTeSQuE proceedings.
When accepted, the 2 pages will be made available on the website of the workshop.

We expect industrial practitioners to submit the summaries on the use of machine learning for software quality, with the purpose of stimulating the discussion, initiate challenging new avenues, etc.

All papers should be submitted in the PDF format, conforming to the ACM conferences template, through HotCRP.com.
To attract participants for this workshop, we will distribute its call for paper in mailing lists related to the communities of software engineering and machine learning.
The desired number of participants ranges from 20-25 participants.
Selected papers will be invited to be extended in a special issue of a well-established journal in the field of software engineering.

\subsection{Review Process}
The process for paper submission and evaluation will be similar to the ESEC/FSE one.
Therefore, all submitted papers will undergo a rigorous peer review process, with emphasis on their originality, quality, soundness and relevance.
In the same way as ESEC/FSE, the workshop will follow a double-blind review process, where three program committee members will review the submitted papers.
Afterwards, the program committee will jointly make the final decision concerning the acceptance of individual papers, based on the reviews.

\subsection{Expected Dates}
For the expected dates for submission, we conform to the ESEC/FSE 2020 indications:
\begin{itemize}[itemsep=0.5em]
	\item Submission July 8$^{th}$, 2020
	\item Notification Aug 26$^{th}$, 2020
\end{itemize}

\subsection{Confirmed Program Committee Members}
The program committee members are selected among both senior and junior researchers working on the topics of the workshop with the aim of (a) ensuring a high review quality, (b) supporting the emergence of junior researchers in the community and (c) attract established researchers from communities related to artificial intelligence and machine learning.
The program committee members are:
\gc{I wrote "c" for the confirmed,this list belongs to the PC of last year, feel free to invite new people}
\begin{itemize}[itemsep=0.5em]
	\item (c) Mathieu Acher, University of Rennes I 
	\item (c) Francesca Arcelli Fontana, Univ. Milano Bicocca 
	\item (c) Apostolos Ampatzoglou, University of Macedonia 
	\item Elvira-Maria Arvanitou, University of Macedonia 
	\item Earl T. Barr, University College London 
	\item Stamatia Bibi, University of Western Macedonia 
	\item Jordi Cabot, Open University of Catalonia 
	\item Alexander Chatzigeorgiou, University of Macedonia 
	\item Jurgen Cito, Wien University of Technology 
	\item Eleni Constantinou, Eindhoven University of Technology 
	\item Maxime Cordy, University of Luxembourg 
	\item Steve Counsell, Brunell University 
	\item Jesse Davis, Katholieke Universiteit Leuven 
	\item Xavier Devroey, Delft University of Technology 
	\item (c) Dario Di Nucci,Jheronimus Academy of Data Science 
	\item Rémi Emonet, Laboratoire Hubert Curien 
	\item Daniel Feitosa, University of Groningen 
	\item Benoit Frenay, University of Namur 
	\item Suman Jana, Columbia University
	\item George Kakarontzas, Tech. Educ. Inst. of Thessaly
	\item Marta Kwiatkowska, University of Oxford
	\item Lech Madeyski, Wroclaw University of Technology
	\item Karl Meinke, KTH Royal Institute of Technology
	\item Tim Menzies, NC State University
	\item Mirosław Ochodek, Poznan University
	\item Haidar Osman, University of Bern
	\item (c) Fabio Palomba, University of Salerno
	\item Annibale Panichella, Delft University of Technology
	\item Sebastiano Panichella, University of Zurich
	\item (c) Gilles Perrouin, University of Namur
	\item Jean-Francois Raskin, University Libre de Bruxelles
	\item Koushik Sen, University of California, Berkeley
	\item Alyson Smith, Decisive Analytics Corporation
	\item Davide Taibi, Free University of Bozen
	\item (c) Damian A. Tamburri, Jheronimus Academy of Data Science
	\item Paolo Tonella, University della Svizzera italiana
	\item Yves Le Traon, University of Luxembourg
	\item (c) Bartosz Walter, Poznan University of Technology
	\item (c) Maxime Cordy, University of Luxembourg
	\item Aiko Yamashita, Oslo University
	\item (c) Valerio Terragni, USI Università della Svizzera italiana
	\item (c) Gemma Catolino, Delft University of Technology
	\item (c) Pasquale Salza, University of Zurich
	\item (c) Foutse Khom, Polytechnique Montréal
\end{itemize}

%!TEX root = main.tex

\section{Publicity Plans}

As for the publicity part, we already have an active Twitter account - \url{https://twitter.com/MaLTeSQuE_2019} - that counts more than 120 followers and a Facebook Page - \url{https://www.facebook.com/maltesque/} - with more than 40 followers.
Moreover, at this link you can reach the website of the last edition of MaLTeSQuE 2019 - \url{https://maltesque2019.github.io}.

In the following we indicate our additional plans.

\subsection{Participants Solicitation}

We expect a number of about 40 participants, considering the 32 people who attended the edition of the last year, when the conference was held in Tallin, co-located with ESEC/FSE 2019.
The workshop will be open to all ESEC/FSE participants who will be registered for the same days of the workshop.

Furthermore, we plan to support the wide dissemination of the accepted contributions as well as the participants' discussion.
To this purpose, we will ask the contributors to share their presentation on our website and we aim at being highly present on social networks, such as Facebook and Twitter.
Thus, we plan to have a person responsible for publicity and dissemination of the contents during the entire workshop.

\subsection{Workshop Description for Publicity}
%At this link you can reach the website of the last edition of MaLTeSQuE 2019
%\url{https://maltesque2019.github.io}

\begin{myquote}{0.2in}
    \itshape
    The assessment of software quality is one of the most multifaceted (e.g., structural quality, product quality, process quality, etc.) and subjective aspects of software engineering (since in many cases it is substantially based on expert judgement).
    Such assessments can be performed at almost all phases of software development (from project inception to maintenance) and at different levels of granularity (from source code to architecture). However, human judgement is: (a) inherently biased by implicit, subjective criteria applied in the evaluation process, and (b) its economical effectiveness is limited compared to automated or semi-automated approaches.
    To this end, researchers are still looking for new, more effective methods of assessing various qualitative characteristics of software systems and the related processes.
    In recent years we have been observing a rising interest in adopting various approaches to exploiting machine learning (ML) and automated decision-making processes in several areas of software engineering. These models and algorithms help to reduce effort and risk related to human judgment in favor of automated systems, which are able to make informed decisions based on available data and evaluated with objective criteria.
    Thus, the adoption of machine learning techniques seems to be one of the most promising ways to improve software quality evaluation.
    Conversely, learning capabilities are increasingly often embedded within software, including in critical domains such as automotive and health.
    This calls for the application of quality assurance techniques to ensure the reliable engineering of ML-based software systems.
\end{myquote}

%!TEX root = main.tex

\section{Follow-Up Plans}
We are also planning to reserve time during the workshop in order to open a discussion with all participants. This discussion aims at producing, as tangible outcome, a manuscript that describes the grand challenges and roadmap for future research in the field of machine learning for software quality evaluation. For instance, a possible grand challenge could be the responsibility of machine learning in society and research (\eg when it is useful and when not) as well as which could be new areas in software engineering where machine learning techniques could be exploited or re-adapted. The produced manuscript will be submitted as a vision paper to a conference or journal.

Finally, we plan to invite the workshop papers to extend their work in a special issue of a well-established journal in the field of software engineering. Last year, we opened a collaboration with the Journal of Software and Systems (JSS), edited by Elsevier. The last year's special issue is still undergoing, yet it received more than 20 submissions. 

%!TEX root = main.tex
% !TeX spellcheck = en_US

\section{Organizers}

The organizers are all researchers with a software engineering and machine learning background.

\label{sec:organizers}
\subsection{Gemma Catolino}
Dr. Gemma Catolino is a post-doctoral researcher at the Tilburg University $-$ Jheronimus Academy of Data Science, the Netherlands.
In March 2020, she received the European Ph.D. Degree from the University of Salerno, Italy. Her research interests include human factors in software maintenance and evolution, empirical software engineering, source code quality, and mining software repositories.
She received the best Master's thesis award from the Italian Software Metrics Association in 2016, and a gold medal for the Microsoft Student Research Competition in 2018. She serves and has served as a program committee member of various international conferences (ICPC, MSR, ICSME) and as a referee for flagship journals in the fields of software engineering (EMSE, JSS, TSE). She co-organized the 4th edition of the IEEE Virtual Workshop on Machine Learning Techniques for Software Quality Evaluation 2020 (MaLTeSQuE) collocated with ESEC/FSE 2020 and also co-organising the 4th International Workshop on Software Health in Projects, Ecosystems and Communities 2021(SoHeal) collocated with ICSE 2021.

\medskip
\noindent Contact her at: \href{mailto:g.catolino@tilburguniversity.edu}{g.catolino@tilburguniversity.edu}.\\
More info at: \url{https://www.gemmacatolino.com}

\subsection{Valentina Lenarduzzi}
Dr. Valentina Lenarduzzi is a post-doctoral researcher at LUT University , Finland. 
She obtained the Ph.D. in Computer Science at the University of Insubria (Italy) in 2015, working on data analysis in Software Engineering. She also spent 8 months as Visiting Researcher at the Technical University of Kaiserslautern and Fraunhofer Institute for Experimental Software Engineering (IESE) working on Empirical Software Engineering in Embedded Software and Agile projects.
She was postdoctoral researcher at Tampere University (Finland), and the Free University of Bozen-Bolzano, (Italy). In 2011 She was one of the co-founders of Opensoftengineering s.r.l., a spin-off company of the University of Insubria.
Her research activities are related to data analysis in software engineering, software quality, software maintenance and evolution, focusing on Technical Debt, Code and Architectural smells. 

\medskip
\noindent 
Contact her at: \href{mailto:valentina.lenarduzzi@lut.fi}{valentina.lenarduzzi@lut.fi}.\\
More info at: \url{http://www.valentinalenarduzzi.it}

\subsection{Daniel Feitosa}
Dr. Daniel Feitosa is an Assistant Professor at Campus Fryslân of the University of Groningen, the Netherlands. He is also the Chief Data Scientist at the Data Research Centre, and a guest researcher in the Faculty of Science and Engineering of the University of Groningen. 
Dr. Feitosa holds a B.Sc. (2010) and M.Sc. (2013) in Computer Science from the University of São Paulo, Brazil, and a Ph.D. (2019) in Software Engineering by the University of Groningen. In 2020, he was nominated for the GEC (Groningen Engineering Center) Best PhD Thesis award. Dominant themes in the work of Dr. Feitosa include software quality, empirical software engineering, software architecture and embedded systems.
He has served in the organization committee of the 46th Euromicro Conference on Software Engineering and Advanced Applications, and is co-organizer of the 4th International Conference on Technical Debt (collocated with ICSE 2021). Dr. Feitosa has also acted as a referee for prominent journals in the field of software engineering (IST, JSS, TSE), and he has served in the programme committee of internal conferences (SEAA, ECSA) as well as previous editions of MaLTeSQuE.

\medskip
\noindent Contact him at: \href{mailto:d.feitosa@rug.nl}{d.feitosa@rug.nl}.\\
More info at: \url{https://feitosa-daniel.github.io/}\\

\subsection{Apostolos Ampatzoglou}
Dr. Apostolos Ampatzoglou is an Assistant Professor in the Department of Applied Informatics in University of Macedonia (Greece), where he carries out research and teaching in the area of software engineering. Before joining University of Macedonia he was an Assistant Professor in the University of Groningen (Netherlands). He holds a BSc on Information Systems (2003), an MSc on Computer Systems (2005) and a PhD in Software Engineering by the Aristotle University of Thessaloniki (2012). He has published more than 100 articles in international journals and conferences, and is/was involved in over 15 R\&D ICT projects, with funding from national and international organizations. Based on these performance indicators, he has been nominated as the 3rd most active Early-Stage Researcher in software engineering domain for the period 2010-2017. His current research interests are focused on technical debt management, software maintainability, game engineering, software quality management, open source software, and software design. He has served in the organization of various conferences and workshops from all possible roles, and is a member of the steering committee of Maltesque. Dr. Ampatzoglou has edited special issues in the Journal of Systems and Software, and Journal of Software: Evolution and Process.

\medskip
\noindent Contact him at: \href{mailto:a.ampatzoglou@uom.edu.gr}{a.ampatzoglou@uom.edu.gr}.\\
More info at: \url{https://users.uom.gr/~a.ampatzoglou/#social}\\
\textbf{He acts as the main contact.}


\balance
	
\end{document}
\endinput
