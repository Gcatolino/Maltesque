%!TEX root = main.tex
% !TeX spellcheck = en_US

\section{Organizers}

The organizers are all researchers with a software engineering and machine learning background.

\label{sec:organizers}
\subsection{Gemma Catolino}
Dr. Gemma Catolino is a post-doctoral researcher at the Tilburg University $-$ Jheronimus Academy of Data Science, the Netherlands.
In March 2020, she received the European Ph.D. Degree from the University of Salerno, Italy. Her research interests include human factors in software maintenance and evolution, empirical software engineering, source code quality, and mining software repositories.
She received the best Master's thesis award from the Italian Software Metrics Association in 2016, and a gold medal for the Microsoft Student Research Competition in 2018. She serves and has served as a program committee member of various international conferences (ICPC, MSR, ICSME) and as a referee for flagship journals in the fields of software engineering (EMSE, JSS, TSE). She co-organized the 4th edition of the IEEE Virtual Workshop on Machine Learning Techniques for Software Quality Evaluation 2020 (MaLTeSQuE) collocated with ESEC/FSE 2020 and also co-organising the 4th International Workshop on Software Health in Projects, Ecosystems and Communities 2021(SoHeal) collocated with ICSE 2021.

\medskip
\noindent Contact her at: \href{mailto:g.catolino@tilburguniversity.edu}{g.catolino@tilburguniversity.edu}.\\
More info at: \url{https://www.gemmacatolino.com}


\subsection{Valentina Lenarduzzi}


\medskip
\noindent 
Contact him at: \href{mailto:valentina.lenarduzzi@lut.fi}{valentina.lenarduzzi@lut.fi}.\\
More info at: \url{http://www.valentinalenarduzzi.it}

\subsection{Daniel Feitosa}
Dr. Daniel Feitosa is an Assistant Professor at Campus Fryslân of the University of Groningen, the Netherlands. He is also the Chief Data Scientist at the Data Research Centre, and a guest researcher in the Faculty of Science and Engineering of the University of Groningen. 
Dr. Feitosa holds a B.Sc. (2010) and M.Sc. (2013) in Computer Science from the University of São Paulo, Brazil, and a Ph.D. (2019) in Software Engineering by the University of Groningen. In 2020, he was nominated for the GEC (Groningen Engineering Center) Best PhD Thesis award. Dominant themes in the work of Dr. Feitosa include software quality, empirical software engineering, software architecture and embedded systems.
He has served in the organization committee of the 46th Euromicro Conference on Software Engineering and Advanced Applications, and is co-organizer of the 4th International Conference on Technical Debt (collocated with ICSE 2021). Dr. Feitosa has also acted as a referee for prominent journals in the field of software engineering (IST, JSS, TSE), and he has served in the programme committee of internal conferences (SEAA, ECSA) as well as previous editions of MaLTeSQuE.

\medskip
\noindent Contact him at: \href{mailto:d.feitosa@rug.nl}{d.feitosa@rug.nl}.\\
More info at: \url{https://feitosa-daniel.github.io/}\\


\subsection{Apostolos Ampatzoglou}
Dr. Apostolos Ampatzoglou is an Assistant Professor in the Department of Applied Informatics in University of Macedonia (Greece), where he carries out research and teaching in the area of software engineering. Before joining University of Macedonia he was an Assistant Professor in the University of Groningen (Netherlands). He holds a BSc on Information Systems (2003), an MSc on Computer Systems (2005) and a PhD in Software Engineering by the Aristotle University of Thessaloniki (2012). He has published more than 100 articles in international journals and conferences, and is/was involved in over 15 R&D ICT projects, with funding from national and international organizations. Based on these performance indicators, he has been nominated as the 3rd most active Early-Stage Researcher in software engineering domain for the period 2010-2017. His current research interests are focused on technical debt management, software maintainability, game engineering, software quality management, open source software, and software design. He has served in the organization of various conferences and workshops from all possible roles, and is a member of the steering committee of Maltesque. Dr. Ampatzoglou has edited special issues in the Journal of Systems and Software, and Journal of Software: Evolution and Process.


\medskip
\noindent Contact him at: \href{mailto:a.ampatzoglou@uom.edu.gr}{a.ampatzoglou@uom.edu.gr}.\\
More info at: \url{https://users.uom.gr/~a.ampatzoglou/#social}\\
He acts as the main contact.
