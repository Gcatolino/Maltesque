%!TEX root = main.tex
% !TeX spellcheck = en_US

\section{Workshop Format}
\label{sec:format}

The workshop will follow a one-day format, consisting of 3 to 4 sessions, depending on the number of papers accepted for publication.
The workshop is intended to be highly interactive: for this reason, each accepted paper will have a maximum of 15/20 minutes for the presentation, followed by 10/15 minutes for questions and discussion.
To foster interaction throughout the day, we plan to start the workshop by means of a “turbo mix-and-talk” session. Each workshop attendee talks for two minutes to a random attendee she does not know yet, then switches partner after a signal. We aim for a highly interactive workshop that fosters and promotes collaborations between the participants. %There will be some time set aside to support this during the day.

We plan to invite a keynote speaker who will be selected as a kind of advertisement to raise the public profile of the workshop and attract more people to attend and submit their research. It means that the person chosen will be a well-known personality in that particular topic. 

The previous edition of Maltesque has gone virtual due to the current COVID-19 pandemic. If travel is not going to be possible for some countries, we will be flexible and allow virtual presentations for affected participants, we will also allow virtual presentations for participants preferring to stay at home. If the situation with the COVID-19 pandemic is still the same as today, Maltesque 2021 will go virtual. In that case, the virtual setting of Maltesque 2021 will make use of two platforms: Zoom and Slack. Zoom will be used to stream all workshop sessions, while Slack will be used for Q\&A during sessions and coffee breaks. All papers' authors will be required to pre-record the presentations so that the session chair can live-stream them during the workshop. Afterward, the authors will be engaged in Q\&A sessions. Participants can interact by posting their questions on the dedicated Slack channels, and upvote questions if they like them or have the same questions. The session chairs will then ask the questions out loud so that the author can answer. We will also consider other solutions proposed by ESEC/FSE in case of going virtual.
Since participants will join Maltesque 2021 from different time zones, the workshop program will allow all presenters to present at a reasonable hour based on their time zone. This setting will allow for a broader participation. In addition, we will record each presented paper, with the prior consent of the authors, and we will upload them on our website giving the possibility to all the authors from different time zone to watch all the presentations.