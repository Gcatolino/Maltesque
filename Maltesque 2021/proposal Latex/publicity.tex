%!TEX root = main.tex
% !TeX spellcheck = en_US

\section{Publicity Plans}

As for the publicity part, we already have an active Twitter account (\url{https://twitter.com/MaLTeSQuE_2020}) with more than 170 followers, and a Facebook Page (\url{https://www.facebook.com/maltesque/}) with more than 50 followers.
Moreover, at this link you can reach the website of the last edition of MaLTeSQuE 2020: \url{https://maltesque2020.github.io}.
In the following we indicate our additional plans.

\subsection{Participants Solicitation}

We expect a number of about 40 participants, considering that more tha 40 people attended virtually the edition of the last year (ESEC/FSE 2020).
The workshop will be open to all ESEC/FSE participants who will be registered for the same days of the workshop.

Furthermore, we plan to support the wide dissemination of the accepted contributions as well as the participants' discussion.
To this purpose, we will ask the contributors to share their presentation on our website and we aim at being highly present on social networks, such as Facebook and Twitter.
Thus, we plan to have a person responsible for publicity and dissemination of the contents during the entire workshop.

\subsection{Workshop Description for Publicity}
 \textit{The assessment of software quality is one of the most multifaceted (e.g., structural quality, product quality, process quality, etc.) and subjective aspects of software engineering (since in many cases it is substantially based on expert judgement).
    Such assessments can be performed at almost all phases of software development (from project inception to maintenance) and at different levels of granularity (from source code to architecture). However, human judgement is: (a) inherently biased by implicit, subjective criteria applied in the evaluation process, and (b) its economical effectiveness is limited compared to automated or semi-automated approaches.
    To this end, researchers are still looking for new, more effective methods of assessing various qualitative characteristics of software systems and the related processes.
    In recent years we have been observing a rising interest in adopting various approaches to exploiting machine learning (ML) and automated decision-making processes in several areas of software engineering. These models and algorithms help to reduce effort and risk related to human judgment in favor of automated systems, which are able to make informed decisions based on available data and evaluated with objective criteria.
    Thus, the adoption of machine learning techniques seems to be one of the most promising ways to improve software quality evaluation.
    Conversely, learning capabilities are increasingly often embedded within software, including in critical domains such as automotive and health.
    This calls for the application of quality assurance techniques to ensure the reliable engineering of ML-based software systems.}