%!TEX root = main.tex
% !TeX spellcheck = en_US

\section{Introduction}
\label{sec:motivation}

The assessment of software quality is one of the most multifaceted (\eg structural, product, and quality) and subjective aspects of software engineering (since in many cases it is substantially based on expert judgment).
Such assessments can be performed at almost all the phases of software development (from project inception to maintenance) and at different levels of granularity (from source code to architecture).
However, human judgment is inherently biased by implicit, subjective criteria applied to the evaluation process, and its economical effectiveness is limited compared to automated or semi-automated approaches.
To this end, researchers are still looking for new and more effective methods for assessing various qualitative characteristics of software systems and the related processes.
In recent years, we have been observing an increasing interest in adopting various approaches to exploiting machine learning and automated decision-making processes in several areas of software engineering.
The machine learning models and algorithms help to reduce effort and risk related to human judgment in favor of automated systems, which are able to make informed decisions based on available data and evaluated with objective criteria.
Therefore, the adoption of machine learning techniques seems to be one of the most promising ways to improve software quality evaluation.
Conversely, learning capabilities are increasingly embedded within software, including in critical domains such as automotive and health.
For this reason, the application of quality assurance techniques is required to ensure the reliable engineering of software systems based on machine learning.
After the successful past editions of the workshop on Machine Learning Techniques for Software Quality Evaluation, which have been held in:
\begin{enumerate}
    \item Klagenfurt (Austria) on February \nth{21}, 2017, collocated with SANER 2017,
    \item Campobasso (Italy) on March \nth{23} , 2018, collocated with SANER 2018, and
    \item Tallin (Estonia) on August \nth{27} 2019, collocated with ESEC/FSE 2019,
    \item Virtual Event, November  \nth{16} 2020, collocated with ESEC/FSE 2020,
\end{enumerate}
we propose a novel edition of the workshop.
We aim at reaching the same, or hopefully better result, the workshop had the last year more than of 40 participants.
