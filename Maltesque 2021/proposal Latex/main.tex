% !TeX spellcheck = en_US

\documentclass[sigconf]{acmart}

\settopmatter{printfolios=true}
\acmConference[ESEC/FSE 2021]{The 29th ACM Joint European Software Engineering Conference and Symposium on the Foundations of Software Engineering}{23 - 27 August, 2021}{Athens, Greece}
\acmISBN{}
\makeatletter
\renewcommand\@formatdoi[1]{\ignorespaces}
\makeatother

\usepackage{enumitem}
\usepackage{xspace}
\usepackage{balance}
\usepackage{amsmath}
\usepackage[super]{nth}

\newcommand{\ie}{i.e.,\@\xspace}
\newcommand{\eg}{e.g.,\@\xspace}
\newcommand{\etc}{etc.\@\xspace}
\newcommand{\etal}{et al.\@\xspace}

\newenvironment{myquote}[1]%
  {\list{}{\leftmargin=#1\rightmargin=#1}\item[]}%
  {\endlist}

\begin{document}

\title{MaLTeSQuE: Machine Learning Techniques\\for Software Quality Evaluation}
\subtitle{A workshop proposal for ESEC/FSE 2021 -- \url{https://maltesque2020.github.io}}

%!TEX root = main.tex
% !TeX spellcheck = en_US

\author{Gemma Catolino}
\affiliation{%
    \institution{Delft University of Technology}
    \city{Delft}
    \state{The Netherlands}
}
\email{g.catolino@tudelft.nl}

\author{Pasquale Salza}
\affiliation{%
    \institution{University of Zurich}
    \city{Zurich}
    \state{Switzerland}
}
\email{salza@ifi.uzh.ch}

\author{Foutse Khomh}
\affiliation{%
    \institution{Polytechnique Montréal}
    \city{Montréal}
    \state{Canada}
}
\email{foutse.khomh@polymtl.ca}


\begin{abstract}
We propose the 5$^{th}$ edition of the workshop on Machine Learning Techniques for Software Quality Evaluation (MaLTeSQuE) to be collocated with the next edition of ESEC/FSE 2021.
We aim at creating an enthusiastic forum for researchers and practitioners to present and discuss new ideas, trends and results concerning the application of machine learning methods for software quality evaluation, and the application of software engineering techniques to self-learning systems.
\end{abstract}

\keywords{Software Engineering, Machine Learning, Software Quality}

\maketitle

%!TEX root = main.tex
\section{Motivation}
\label{sec:motivation}
The assessment of software quality is one of the most multifaceted (\eg structural quality, product quality, process quality, etc.) and subjective aspects of software engineering (since in many cases it is substantially based on expert judgement). Such assessments can be performed at almost all phases of software development (from project inception to maintenance) and at different levels of granularity (from source code to architecture). However, human judgement is: (a) inherently biased by implicit, subjective criteria applied in the evaluation process, and (b) its economical effectiveness is limited compared to automated or semi-automated approaches. To this end, researchers are still looking for new, more effective methods of assessing various qualitative characteristics of software systems and the related processes.
In recent years we have been observing a rising interest in adopting various approaches to exploiting machine learning (ML) and automated decision- making processes in several areas of software engineering. These models and algorithms help to reduce effort and risk related to human judgment in favor of automated systems, which are able to make informed decisions based on available data and evaluated with objective criteria. Thus, the adoption of machine learning techniques seems to be one of the most promising ways to improve software quality evaluation.
Conversely, learning capabilities are increasingly often embedded within software, including in critical domains such as automotive and health. This calls for the application of quality assurance techniques to ensure the reliable engineering of ML-based software systems.
After the successful editions of the workshop on Machine Learning Techniques for Software Quality Evaluation (MaLTeSQuE), that have been held in Klagenfurt (Austria) on February 21$^{st}$, 2017, collocated with SANER 2017 in Campobasso (Italy) on March 23$^{rd}$ , 2018, collocated with SANER 2018, and in Tallin (Estonia) on August 27$^{th}$ 2019, collocated with ESEC/FSE 2019, we propose a novel edition of the workshop.







%!TEX root = main.tex
\section{Objective and Topics}
\label{sec:objective}
\subsection{Objective}
The aim of the workshop is to provide a forum for researchers and practitioners to present and discuss new ideas, trends and results concerning the application of ML methods to software quality evaluation and the application of software engineering techniques to self-learning systems. We expect that the workshop will help in (1) the validation of existing ML methods for software quality evaluation as well as their application to novel contexts, (2) the effectiveness evaluation of ML methods, both compared to other automated approaches and the human judgement, (3) the adaptation of ML approaches already used in other areas of science in the context of software quality, (4) the design of new techniques to validate ML- based software, inspired by traditional software engineering techniques.
\subsection{Topics}
Topics of interest include, but are not limited to:
\begin{itemize}
\item Application of machine-learning in software quality evaluation,\smallskip
\item Analysis of multi-source data,\smallskip
\item Knowledge acquisition from software repositories,\smallskip
\item Adoption and validation of machine learning models and algorithms in software quality,\smallskip
\item Decision support and analysis in software quality,\smallskip
\item Prediction models to support software quality evaluation,\smallskip
\item Validation and verification of systems, learning,\smallskip
\item Automated machine learning,\smallskip
\item Design of safety-critical learning software,\smallskip
\item Integration of learning systems in software ecosystems.
\end{itemize}
%!TEX root = main.tex
\section{Workshop Format}
\label{sec:format}

The workshop will follow a one-day format, consisting of 3 to 4 sessions, depending on the number of papers accepted for publication.
The workshop is intended to be highly interactive: for this reason, each accepted paper will have a maximum of 15/20 minutes for the presentation, followed by 10/15 minutes for questions and discussion.
We hope that the workshop will foster and promote collaboration, and there will be some time set aside to support this.

We also plan to invite a keynote speaker who will be selected as a kind of advertisement to raise the public profile of the workshop and attract more people to attend and submit their research.
It means that the person chosen will be a well-known personality in that particular topic.

Finally, the accepted papers will be part of the ESEC/FSE proceedings and available for participants in advance through a workshop webpage and in the ACM digital library.

%!TEX root = main.tex
\section{Submission}
\label{sec:submissions}
We are looking for two different types of papers; the first one, \ie technical papers, where we invite to submit original research (even at early stages of evaluation) on how machine learning and software quality assurance can support each other. Papers must not exceed four 6 pages for the main text, including figures, tables, appendices, and references. They will be part of the ESEC/FSE proceedings and available for participants in advance through a workshop webpage. The second one, \ie presentation abstracts, that have to report on research results that have already been published, or that are ready to be submitted to a conference or a journal. They will only be reviewed for relevance, and will not be included in the MaLTeSQuE proceedings. When accepted, the 2 pages will be made available on the website of the workshop. Of particular interest are summaries that try to provoke discussion, initiate challenging new avenues, etc. submitted by industrial practitioners on the use of machine learning for software quality. All papers should be submitted in PDF format (conforming to the ACM conferences template) through HotCRP.com.
To attract participants for this workshop, we will distribute its CfP in mailing lists related to the communities of software engineering and machine learning. The desired number of participants ranges from 20-25 participants.
Selected papers will be invited to be extended in a special issue of a well-established journal in the field of software engineering.
\subsection{Review Process}
The process for paper submission and evaluation will be similar to the ESEC/FSE one. Therefore, all submitted papers will undergo a rigorous peer review process, with emphasis on their originality, quality, soundness and relevance. Like ESEC/FSE, the workshop will follow a double-blind review process, where three PC members will review the submitted papers. Afterwards, the Program Committee will jointly make the final decision concerning acceptance of individual papers, based on the reviews.
\subsection{Program Committee (not yet definitive)}
The Program Committee members are selected among both senior and junior researchers working on the topics of the workshop with the aim of (a) ensuring a high review quality, (b) supporting the emergence of junior researchers in the community and (c) attract established researchers from communities related to artificial intelligence and ML. The Program Committee members are:
\gc{I wrote "c" for the confirmed,this list belongs to the PC of last year, feel free to invite new people}
\begin{itemize}
	\item Mathieu Acher, University of Rennes I
c \smallskip
	\item Francesca Arcelli Fontana, Univ. Milano Bicocca c \smallskip
	\item Apostolos Ampatzoglou, University of Macedonia
c \smallskip
	\item Elvira-Maria Arvanitou, University of Macedonia
\smallskip
	\item Earl T. Barr, University College London
\smallskip
	\item Stamatia Bibi, University of Western Macedonia
\smallskip
	\item Jordi Cabot, Open University of Catalonia
\smallskip
	\item Alexander Chatzigeorgiou, University of Macedonia
\smallskip
	\item Jurgen Cito, Wien University of Technology
\smallskip
	\item Eleni Constantinou, Eindhoven University of Technology \smallskip
	\item Maxime Cordy, University of Luxembourg
\smallskip
	\item Steve Counsell, Brunell University
\smallskip
	\item Jesse Davis, Katholieke Universiteit Leuven
\smallskip
	\item Xavier Devroey, Delft University of Technology
\smallskip
	\item Dario Di Nucci,Jheronimus Academy of Data Science
c \smallskip
	\item Rémi Emonet, Laboratoire Hubert Curien
\smallskip
	\item Daniel Feitosa, University of Groningen
\smallskip
	\item Benoit Frenay, University of Namur
\smallskip
	\item Suman Jana, Columbia University
\smallskip
	\item George Kakarontzas, Tech. Educ. Inst. of Thessaly
\smallskip
	\item Marta Kwiatkowska, University of Oxford
\smallskip
	\item Lech Madeyski, Wroclaw University of Technology
\smallskip
	\item Karl Meinke, KTH Royal Institute of Technology
\smallskip
	\item Tim Menzies , NC State University
\smallskip
	\item Mirosław Ochodek, Poznan University
\smallskip
	\item Haidar Osman, University of Bern
\smallskip
	\item Fabio Palomba, University of Salerno
c \smallskip
	\item Annibale Panichella, Delft University of Technology
\smallskip
	\item Sebastiano Panichella, University of Zurich
\smallskip
	\item Gilles Perrouin, University of Namur
c \smallskip
	\item Jean-Francois Raskin, University Libre de Bruxelles
\smallskip
	\item Koushik Sen, University of California, Berkeley
\smallskip
	\item Alyson Smith, Decisive Analytics Corporation
\smallskip
	\item Davide Taibi, Free University of Bozen
\smallskip
	\item Damian A. Tamburri, Jheronimus Academy of Data Science
c \smallskip
	\item Paolo Tonella, University della Svizzera Italiana
\smallskip
	\item Yves Le Traon, University of Luxembourg
\smallskip
	\item Bartosz Walter, Poznan University of Technology
c \smallskip
	\item Maxime Cordy,University of Luxembourg
c \smallskip
	\item Aiko Yamashita, Oslo University
\smallskip
	\item Gemma Catolino,Delft University of Technology
c \smallskip
	\item Pasquale Salza, University of Zurich
c \smallskip
	\item Foutse Khom, Polytechnique Montréal c \smallskip
\end{itemize}
%!TEX root = main.tex

\section{Publicity Plans}

As for the publicity part, we already have an active Twitter account - \url{https://twitter.com/MaLTeSQuE_2019} - that counts more than 120 followers and a Facebook Page - \url{https://www.facebook.com/maltesque/} - with more than 40 followers.
Moreover, at this link you can reach the website of the last edition of MaLTeSQuE 2019 - \url{https://maltesque2019.github.io}.

In the following we indicate our additional plans.

\subsection{Participants Solicitation}

We expect a number of about 40 participants, considering the 32 people who attended the edition of the last year, when the conference was held in Tallin, colocated with ESEC/FSE 2019.
The workshop will be open to all ESEC/FSE participants who will be registered for the same days of the workshop.

Furthermore, we plan to support the wide dissemination of the accepted contributions as well as the participants’ discussion.
To this purpose, we will ask the contributors to share their presentation on our website and we aim at being highly present on social networks, such as Facebook and Twitter.
Thus, we plan to have a person responsible for publicity and dissemination of the contents during the entire workshop.

\subsection{Workshop Description for Publicity}
%At this link you can reach the website of the last edition of MaLTeSQuE 2019
%\url{https://maltesque2019.github.io}

\begin{myquote}{0.2in}
    \itshape
    The assessment of software quality is one of the most multifaceted (e.g., structural quality, product quality, process quality, etc.) and subjective aspects of software engineering (since in many cases it is substantially based on expert judgement).
    Such assessments can be performed at almost all phases of software development (from project inception to maintenance) and at different levels of granularity (from source code to architecture). However, human judgement is: (a) inherently biased by implicit, subjective criteria applied in the evaluation process, and (b) its economical effectiveness is limited compared to automated or semi-automated approaches.
    To this end, researchers are still looking for new, more effective methods of assessing various qualitative characteristics of software systems and the related processes.
    In recent years we have been observing a rising interest in adopting various approaches to exploiting machine learning (ML) and automated decision-making processes in several areas of software engineering. These models and algorithms help to reduce effort and risk related to human judgment in favor of automated systems, which are able to make informed decisions based on available data and evaluated with objective criteria.
    Thus, the adoption of machine learning techniques seems to be one of the most promising ways to improve software quality evaluation. 
    Conversely, learning capabilities are increasingly often embedded within software, including in critical domains such as automotive and health.
    This calls for the application of quality assurance techniques to ensure the reliable engineering of ML-based software systems.
\end{myquote}

%!TEX root = main.tex

\section{Follow-Up Plans}

TODO.


\balance
%!TEX root = main.tex
% !TeX spellcheck = en_US

\section{Organizers}

The organizers are all researchers with a software engineering and machine learning background.

\label{sec:organizers}
\subsection{Gemma Catolino}
Dr. Gemma Catolino is a post-doctoral researcher at the Tilburg University $-$ Jheronimus Academy of Data Science, the Netherlands.
In March 2020, she received the European Ph.D. Degree from the University of Salerno, Italy. Her research interests include human factors in software maintenance and evolution, empirical software engineering, source code quality, and mining software repositories.
She received the best Master's thesis award from the Italian Software Metrics Association in 2016, and a gold medal for the Microsoft Student Research Competition in 2018. She serves and has served as a program committee member of various international conferences (ICPC, MSR, ICSME) and as a referee for flagship journals in the fields of software engineering (EMSE, JSS, TSE). She co-organize the 4th edition of the IEEE Virtual Workshop on Machine Learning Techniques for Software Quality Evaluation 2020 (MaLTeSQuE) collocated with ESEC/FSE 2020 and also co-organising the 4th International Workshop on Software Health in Projects, Ecosystems and Communities 2021(SoHeal) collocated with ICSE 2021.

\medskip
\noindent Contact her at: \href{mailto:g.catolino@tilburguniversity.edul}{g.catolino@tilburguniversity.edu}.\\
More info at: \url{https://www.gemmacatolino.com}


\subsection{Valentina Lenarduzzi}


\medskip
\noindent 
Contact him at: \href{mailto:valentina.lenarduzzi@lut.fi}{valentina.lenarduzzi@lut.fi}.\\
More info at: \url{http://www.valentinalenarduzzi.it}

\subsection{Daniel Feitosa}


\medskip
\noindent Contact him at: \href{mailto:d.feitosa@rug.nl}{d.feitosa@rug.nl}.\\
More info at: \url{https://www.rug.nl/staff/d.feitosa/}\\


\subsection{Apostolos Ampatzoglou}



\medskip
\noindent Contact him at: \href{mailto:a.ampatzoglou@uom.edu.gr}{a.ampatzoglou@uom.edu.gr}.\\
More info at: \url{https://users.uom.gr/~a.ampatzoglou/#social}\\
He acts as the main contact.
	
\end{document}
