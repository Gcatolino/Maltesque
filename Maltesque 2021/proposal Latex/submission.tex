%!TEX root = main.tex
% !TeX spellcheck = en_US

\section{Paper selection procedure}
\label{sec:submissions}

We are looking for two different types of papers; the first one, \ie \textbf{technical papers}, where we invite to submit original research (even at early stages of evaluation) on how machine learning and software quality assurance can support each other.
Papers must not exceed 6 pages for the main text, including figures, tables, appendices, and references.
They will be part of the ESEC/FSE proceedings under the copyright of the ACM digital library.

The second track is the \ie \textbf{presentation abstracts} which will be up to 2 pages long and will report (i) research results that are either already published or ready to be submitted to software engineering conferences/journal and (ii) industrial talks. This new track aims at stimulating the participation of industrial practitioners - who will be able to present the practices used in their contexts - as well as researchers - who may be interested in receiving feedback from the research community on early ideas. They will only be reviewed for relevance, and will not be included in the MaLTeSQuE proceedings, but the abstracts will be made available on the website of the workshop.
For this track, we expect that participants submit a summary on the use of machine learning for software quality or the validation and verification of systems based on machine learning, with the purpose of stimulating the discussion, initiate challenging new avenues, etc.

All papers should be submitted in the PDF format, conforming to the ACM conferences template, through HotCRP.com.
To attract participants for this workshop, we will distribute its call for paper in mailing lists related to the communities of software engineering and machine learning.

\subsection{Review Process}

The process for paper submission and evaluation will be similar to the ESEC/FSE one.
Therefore, all submitted papers will undergo a rigorous peer-review process, with emphasis on their originality, quality, soundness and relevance.
In the same way as ESEC/FSE, the workshop will follow a double-blind review process, where three program committee members will review each submitted paper.
Afterward, the program committee will jointly decide on the acceptance of individual papers, based on the reviews.

\subsection{Expected Dates}

For the expected dates for submission, we conform to the ESEC/FSE 2021 indications:
\begin{itemize}[topsep=0.5em, itemsep=0.5em]
	\item Submission: May \nth{17}, 2021
	\item Notification: July \nth{1}, 2021
\end{itemize}


\subsection{Confirmed Program Committee Members}
The program committee members are selected among both senior and junior researchers working on the topics of the workshop with the aim of (a) ensuring a high review quality, (b) supporting the emergence of junior researchers in the community and attract established researchers from communities related to artificial intelligence and machine learning.
We are still recruiting program committee members, some of them already accepted the invitation, however, there are other members that nicely served as referees in the last editions of the workshop and we are confident that they will accept again our invitation.

\medskip
\noindent The confirmed program committee members are:
\begin{itemize}[topsep=0.5em, itemsep=0.5em]
  \item Mathieu Acher, University of Rennes I
  \item Eleni Constantinou, Eindhoven University of Technology
  \item Maxime Cordy, University of Luxembourg
  \item Steve Counsell, Brunell University
  \item Francesca Arcelli Fontana, University of Milano Bicocca
  \item Foutse Khomh, Polytechnique Montreal
  \item Francesco Lomio, Tampere University
  \item Rafael Messias Martins, Linnaeus University
  \item Karl Meinke, KTH Royal Institute of Technology
  \item Dario Di Nucci, Jheronimus Academy of Data Science
  \item Mirosław Ochodek, Poznan University
  \item Fabio Palomba, University of Salerno
  \item Gilles Perrouin, University of Namur
  \item Pasquale Salza, University of Zurich
  \item Davide Taibi, Tampere University
  \item Valerio Terragni, USI Università della Svizzera italiana
  \item Dimitrios Tsoukalas, Center for Research and Technology
  \item Bartosz Walter, Poznan University of Technology
\end{itemize}

\smallskip
\noindent while the prospective program committee members are:
\begin{itemize}[topsep=0.5em, itemsep=0.5em]
  \item Jürgen Cito, Vienna University of Technology
  \item Luís Cruz, Delft University of Technology
  \item Giovanni Grano, University of Zurich
  \item Nikolaos Mittas, International Hellenic University
  \item Sebastiano Panichella, ZHAW School of Engineering
  \item Steve Swift, Brunell University
  \item Paul Temple, University of Namur
  \item Zhou Xu, Chongqing University
\end{itemize}
