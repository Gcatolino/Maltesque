%!TEX root = main.tex
\section{Workshop Format}
\label{sec:format}

The workshop will follow a one-day format, consisting of 3 to 4 sessions, depending on the number of papers accepted for publication.
The workshop is intended to be highly interactive: for this reason, each accepted paper will have a maximum of 15/20 minutes for the presentation, followed by 10/15 minutes for questions and discussion.
To foster interaction throughout the day, we plan to start the workshop by means of a “turbo mix-and-talk” session. Each workshop attendee talks for two minutes to a random attendee she does not know yet, then switches partner after a signal. We aim for a highly interactive workshop that fosters and promotes collaborations between the participants. %There will be some time set aside to support this during the day.

We plan to invite a keynote speaker who will be selected as a kind of advertisement to raise the public profile of the workshop and attract more people to attend and submit their research. It means that the person chosen will be a well-known personality in that particular topic. Given that ESEC/FSE 2020 will be held in California, we are very confident that we will be able to select a great keynote speaker.

Finally, the accepted papers will be part of the ESEC/FSE proceedings and available for participants in advance through a workshop webpage and in the ACM digital library.
